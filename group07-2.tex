\documentclass{jarticle}

\usepackage[dvipdfmx]{graphicx}
\usepackage{url}
\usepackage{listings,jlisting}
\usepackage{ascmac}
\usepackage{amsmath,amssymb}
\usepackage{here}

%ここからソースコードの表示に関する設定
\lstset{
    basicstyle={\ttfamily},
    identifierstyle={\small},
    commentstyle={\smallitshape},
    keywordstyle={\small\bfseries},
    ndkeywordstyle={\small},
    stringstyle={\small\ttfamily},
    frame={tb},
    breaklines=true,
    columns=[l]{fullflexible},
    numbers=left,
    xrightmargin=0zw,
    xleftmargin=3zw,
    numberstyle={\scriptsize},
    stepnumber=1,
    numbersep=1zw,
    lineskip=-0.5ex
}
%ここまでソースコードの表示に関する設定 

\title{知能プログラミング演習II 課題4}
\author{グループ07\\
    29114007 池口 弘尚\\
    29114031 大原 拓人\\
    29114048 北原 太一\\
    29114086 飛世 裕貴\\
    29114095 野竹 浩二朗\\
%  {\small (グループレポートの場合は,グループ名および全員の学生番号と氏名が必要)}
}
\date{2020年1月6日}

\begin{document}
\maketitle

\paragraph{提出物} このレポート グループプログラム"group07.zip"
\paragraph{グループ} グループ07
\paragraph{メンバー}
\begin{tabular}{|c|c|c|}
    \hline
    学生番号&氏名&担当箇所\\
    \hline\hline
    29114007&池口弘尚&5-1,5-3\\
    \hline
    29114031&大原拓人&5-4\\
    \hline
    29114048&北原太一&5-1\\
    \hline
    29114086&飛世裕貴&5-2\\
    \hline
    29114095&野竹浩二朗&5-2\\
    \hline
\end{tabular}

\section{課題の説明}
\begin{description}
    \item[必須課題5-1] 目標集合を変えてみたときに,動作が正しくない場合があったかどうか,実行例を示して考察せよ.
    また,もしあったならその箇所を修正し,どのように修正したか記せ.
    \item[必須課題5-2] 教科書のプログラムでは,オペレータ間の競合解消戦略としてランダムなオペレータ選択を採用している.
    これを,効果的な競合解消戦略に改良すべく考察し,実装せよ.
    改良の結果,性能がどの程度向上したかを定量的に(つまり数字で)示すこと.
    \item[必須課題5-3] 上記のプランニングのプログラムでは,ブロックの属性(たとえば色や形など)を考えていないので,色や形などの属性を扱えるようにせよ.ルールとして表現すること.
    例えば色と形の両方を扱えるようにする場合,Aが青い三角形,Bが黄色の四角形,Cが緑の台形であったとする.
    その時,色と形を使ってもゴールを指定できるようにする("green on blue" や"blue on box"のように)
    \item[必須課題5-4] 上記5-2, 5-3で改良したプランニングシステムのGUIを実装せよ.
    ブロック操作の過程をグラフィカルに可視化し,初期状態や目標状態をGUI上で変更できることが望ましい.
    \item[発展課題5-5] ブロックワールド内における物理的制約条件をルールとして表現せよ.
    例えば,三角錐(pyramid)の上には他のブロックを乗せられない等,その世界における物理的な制約を実現せよ.
    \item[発展課題5-6] ユーザが自然言語(日本語や英語など)の命令文によってブロックを操作したり,初期状態/目標状態を変更したりできるようにせよ.
    なお,命令文の動詞や語尾を1つの表現に決め打ちするのではなく,多様な表現を許容できることが望ましい.
    \item[発展課題5-7] 3次元空間 (実世界) の物理的な挙動を考慮したブロックワールドにおけるプランニングを実現せよ.
    なお,物理エンジン等を利用する場合,Java以外の言語のフレームワークを使って実現しても構わない.
    \item[発展課題5-8] 教科書3.3節のプランニング手法を応用できそうなブロック操作以外のタスクをグループで話し合い,新たなプランニング課題を自由に設定せよ.
    さらに,もし可能であれば,その自己設定課題を解くプランニングシステムを実装せよ.
    \item[必須課題6-1] 課題5にやり残した発展課題があれば参考にして拡張しても良いし,全く新しい独自仕様を考案しても構わない.自由に拡張するか,あるいはもし残っていた問題点があれば完成度を高めよ.
\end{description}

\section{課題6-1}
\begin{screen}
課題5にやり残した発展課題があれば参考にして拡張しても良いし,全く新しい独自仕様を考案しても構わない.自由に拡張するか,あるいはもし残っていた問題点があれば完成度を高めよ.
\end{screen}
\subsection{実装29114007-池口弘尚}
私は,前回実装することができなかった以下の3つを実装した.
\begin{itemize}
\item 三角形の上にものが置けないなどの条件付け
\item 具体化時に他の候補も調べる
\item 最適になるようにゴールリストを並べ替える
\end{itemize}

条件付けに関しては,前回実装した禁止制約を利用することによって,容易に実装することができた.
常に働く禁止制約であるtheBlockListに初期状態として
"\#Place ?* on A"
といったようなものを持たせることによって,Aの上にはものを置けないようにすることができる.

この?*は具体化しない文字であり,どのようなものでも一致するようになっている.
これはUnifier中に以下のように実装した.

\begin{lstlisting}[caption=具体化しない変数の実装,label=src:unifier]
boolean varMatching(String vartoken, String token) {
    //どちらかが?*なら具体化せずにマッチング成立
    if(varAll(vartoken)||varAll(token))
        return true;
    //以下省略
}

boolean varAll(String str) {
    // ?*なら具体化しない
    return var(str)&&str.charAt(1)=='*';
}
\end{lstlisting}

前回の実装ではそれぞれの状態が一意に決まっていなければ解を求めることができなかった.
そのため,今回は他の解の候補も調べることができるようにした.

実装は,planningAGoalの一部を変更して,以下のようになった.

\begin{lstlisting}[caption=他の具体化も調べる,label=src:planagoal]
private int planningAGoal(String theGoal, Vector<String> theCurrentState,
        Hashtable<String, String> theBinding, int cPoint, Vector<String> theBlockList) {
    System.out.println("**" + theGoal);
    if (cPoint <= 0) {
        int size = theCurrentState.size();
        //0ではなくcPointから始める
        for (int i = -cPoint; i < size; i++) {
            String aState = theCurrentState.elementAt(i);
            if ((new Unifier()).unify(theGoal, aState, theBinding)) {
                //次のiをマイナスにしたものを返す
                return -i - 1;
            }
        }
    }
    //上で一回でも具体化されていて,ここを通ったら失敗
    if (cPoint < 0)
        return Integer.MIN_VALUE;
    //以下省略
}
\end{lstlisting}

planningAGoalは-1でエラーを返す使用だったが,仕様変更でInteger.MIN\_VALUEでエラーを表すように変更した.
theCurrentStateとの照合では,常に0を返していたところを次のインデックスに変えることによって,全ての候補を調べることが可能になった.
また,一度でも具体化されているものはオペレータを必要とすることはないため,cPointの値でカットするようになっている.
\newline

また,前回までは与えられたゴールの順番の通りに解を導いていたが,その順番によっては同じことを繰り返してしまうことがあるため,最初に並べなおすように変更した.
実装は以下のようになった.

\begin{lstlisting}[caption=ゴールリストの中身を並べなおす,label=src:blockList]
Vector<String> checkRoute(Vector<String> goalList, Vector<String> initState) {
    Vector<String> newGoal = new Vector<>();
    Vector<String> list = new Vector<>();
    for (String goal : goalList) {
        StringTokenizer tokenizer = new StringTokenizer(goal);
        String x = tokenizer.nextToken();
        tokenizer.nextToken();
        String y = tokenizer.nextToken();
        //xとyが名前になるようにする
        for (String state : initState) {
            if ((new Unifier()).unify("?x has a characteristic of " + x, state)) {
                StringTokenizer t = new StringTokenizer(state);
                x = t.nextToken();
            }
            
            if ((new Unifier()).unify("?x has a characteristic of " + y, state)) {
                StringTokenizer t = new StringTokenizer(state);
                y = t.nextToken();
            }
        }
        boolean isadd = false;
        for (int i = 0; i < list.size(); i++) {
            //A on Bの前にB on Cをする
            if (list.get(i).equals(x)) {
                list.add(i, y);
                newGoal.add(i, goal);
                isadd = true;
                break;
            }
        }
        if (!isadd) {
            list.add(y);
            newGoal.add(goal);
        }
    }
    return newGoal;
}
\end{lstlisting}

\subsection{考察29114007-池口弘尚}
条件付けは,応用することによって,様々な特性を付与することができると思われる.

解の候補だが,これによって形や色などを同列に扱って,オペレータを減らすことには成功した.
しかし,rect(B) on rect(C)のようなゴールを設定すると,rectの具体化がBで止まってしまい,B on Bとなってしまう.
これをプランニングで解決しようとすると,より根本的な見直しが必要になると思われ,今回は実装できなかった.

ゴールリストの順番の並べ替えだが,ここに手を加えれば,このタイプの問題ではより効率よく解を求めることができると思われる.
しかし,今回はプランニングの課題であるために,実行可能であるかなどをあえて確かめないようにしている.
私は、目標状態をソートするメソッドsortGoalListの作成を担当した.
\subsection{手法29114048-北原太一}
私は、目標状態をソートするメソッドsortGoalListの作成を担当した.\\
課題5のプログラムにおいて,ソースコード\ref{lst:initGoal}のような目標集合を考える.
\begin{lstlisting}[caption=Planner.java(一部抜粋),label=lst:initGoal]
private Vector<String> initGoalList() {
  Vector<String> goalList = new Vector<String>();
  goalList.addElement("A on B");
  goalList.addElement("B on C");
  return goalList;
}
  \end{lstlisting}
 
この状態で実行すると,
\begin{enumerate}
\item[手順1] AをBの上に乗せる
\item[手順2] AをBの上から降ろす
\item[手順3] BをCの上に乗せる
\item[手順4] AをBの上に乗せる
\end{enumerate}
以上のような手順で問題を解決し,手順1, 2は無駄な操作となってしまう.状態集合にアサーションを追加する順番を入れ替えれば無駄な手順がなくなるため,我々は追加されたアサーションの順番をソートすることにした.

\subsection{実装29114048-北原太一}
実装に際して,directPriorityというHashTableと,indirectPriorityというVectorを考える.

directPriorityは,ブロックとその階数を対応付けている.例えば,テーブルの上にA,Aの上にB,Bの上にCがある場合,Aの階数は0,Bの階数は1,Cの階数は2である.

indirectPriorityは,ブロックとその下にあるブロックの対応を格納している.先述の例に従うと,indirectPriorityには\{B, A\}と\{C, B\}が格納される.
  
sortGoalListメソッドにおいては,まず目標集合に含まれるすべてのアサーションに対し,それが``ontable X''ならばdirectPriorityにXの階数が0であるという対応を追加し,``X on Y''ならばindirectPriorityに\{X, Y\}を追加する.その後,後述する方法でindirectPriorityをdirectPriorityに変換し,階数の少ないブロックに言及するアサーションが最初になるように目標状態をソートする.

\subsubsection{indirectPriorityからdirectPriorityへの変換}
変換は,以下のような手順で行う.
\begin{enumerate}
\item indirectPriorityから要素を1つ取り出す
\item その要素の2つ目のブロック,すなわち,下側にあるブロックの階数がすでにdirectPriorityからわかる場合,1つ目のブロック,すなわち,上側にあるブロックと(下側のブロックの階数+1)の対応をdirectPriorityに格納する
\item 2.の条件が満たされなかった場合,下側にあるブロックの階数を目標集合の要素数,上側にあるブロックの階数をその数+1とし,対応をdirectPriorityに格納する
\end{enumerate}

この手順をindirectPriorityの要素がなくなるまで繰り返すことにより,すべてのブロックの階数が分かる.

\subsection{考察29114048-北原太一}
sortGoalListメソッドをinitGoalListメソッドの直後に呼び出すことにより,効率よく目標アサーションの選択を行うことができるようになった.しかしながら,このメソッド中に二重以上のfor文が3回使われており,計算時間は長くなってしまっているため,プランニング全体の効率はあまり向上したとは言えない.
\subsection{手法29114095-野竹浩二朗}
私は前回担当したオペレータの選択をさらに効率よく選択できるようにした.\\
オペレータに優先度をずっと固定ではなく,その直前に選択したオペレータによってその都度優先度を振りなおす.

\subsection{実装29114095-野竹浩二朗}
優先度を振りなおせるようにするため,以下のメソッドをPlanner.javaに追加した.
\begin{itemize}
\item 直前のオペレータは何かを判別するjudgeName(Operator)
\item 優先度を振りなおすresetPriority(int)
\end{itemize}
また,優先度をどう振ったか見やすくするため,配列priorityを作り,OPERATOR1から順に優先度を格納した.\\
まず,judgeNameメソッドを以下に示す.
\begin{lstlisting}
public int judgeName(Operator op){
	if((op.name).contains("Place")){
		return 1;
	}else if((op.name).contains("remove")){
		return 2;
	}else if((op.name).contains("pick")){
		return 3;
	}else if((op.name).contains("put")){
		return 4;
	}else{
		return 0;
	}
}
\end{lstlisting}
オペレータの名前を取得し,含まれる単語でオペレータを判別している.initOperatorsではオペレータを12個生成しているが,実際に積み木を動かすオペレータは4個であるため,その4個のみを判定している.\\
次にresetPriorityメソッドを以下に示す.
\begin{lstlisting}
void resetPriority(int judgeOperator){
	if(judgeOperator == 1){
		priority[0] = 7;
		priority[1] = 7;
		priority[2] = 2;
		priority[3] = 1;
	}else if(judgeOperator == 2){
		priority[0] = 7;
		priority[1] = 7;
		priority[2] = 1;
		priority[3] = 2;
	}else if(judgeOperator == 3){
		priority[0] = 2;
		priority[1] = 2;
		priority[2] = 7;
		priority[3] = 1;
	}else if(judgeOperator == 4){
		priority[0] = 2;
		priority[1] = 3;
		priority[2] = 1;
		priority[3] = 7;
	}
}
\end{lstlisting}
上記のjudgeNameの返り値を取得によって優先度を振っている.ここでも変更を加えるオペレータを4個に絞って優先度を変更した.
\subsection{実行結果29114095-野竹浩二朗}
出力された結果をすべて載せてしまうと長くなってしまうため,30回実行し,その平均時間のみを示す.また,性能の差を示すため前回の結果も共に示す.
\begin{table}[H]
\begin{center}
\begin{tabular}{|l|c|}
\hline
前回の平均時間(ミリ秒) & 1103 \\ \hline
今回の平均時間(ミリ秒) & 938 \\ \hline
\end{tabular}
\end{center}
\end{table}
平均時間が約15%短くなっていることがわかった.
\subsection{考察29114095-野竹浩二朗}
上述のように,優先度を振りなおしながらプランニングした方が高速に処理ができていることがわかった.\\
処理時間は短くなったが,前回からの問題であるプログラマー依存性の高さは解消することができなかった.また,judgeNameでは名前に含まれる単語をもとに判定を行ったが,これはオペレータの数が少なく含まれる単語が少ないからこそできた方法である.オペレータが増えた場合でも判定する方法として,今回の課題でも使っているUnifierを用いて判定することができると考えた.ただしUnifierを通すとさらにそこで処理に時間がかかってしまうと思い,この課題では単純な方法で判定を行った.
\subsection{実行結果29114086-飛世裕貴}
ここでは改良した解消戦略を用いたプランニングを30回実行し,その平均実行時間を示す.また,性能の差を示すため前回実装した解消戦略を利用した際の平均時間も共に示す.
\begin{table}[H]
\begin{center}
\begin{tabular}{|l|c|}
\hline
戦略 & 平均実行時間(ミリ秒) \\ 
\hline
前回の解消戦略 & 1098 \\ \hline
今回の解消戦略 & 921 \\ \hline
\end{tabular}
\end{center}
\end{table}
この結果より,平均時間が約16%短くなっていることがわかった.
\subsection{考察29114086-飛世裕貴}
上記したように,直前に選択したオペレータによって優先度を再設定しながらプランニングを実行した方がより高速に処理ができることがわかった.しかし,前回も取り上げたプログラマー依存性の問題は解消することができなかった.この問題の改善方法として,オペレータの出現頻度を用いる方法があげられる.各オペレータの出現頻度を正規化し,その値の大小関係にしたがって優先度を与える.このようにすることでプログラマが優先度を設定することなく,プランニングが可能になると考えられる.またこの方法に加えて,今回のように直前に選択されたオペレータごとに初期値を設定することで,プログラマの既存知識を利用することも可能になると考える.

\section{感想29114007-池口弘尚}
まだまだ正確に解くことができない組み合わせが存在している.
それらを,プランニングを用いて上手く解く方法が思い浮かばなかったため,実装することができなかった.
特に色や形などは,最初に一括で変えてしまえば楽に解くことができるが,それではプランニングにはならないと思われたため,今回のような実装になった.
しかし,この実装方法のために,具体化の候補が複数になり,問題がより複雑になり,上手く解く方法がわからなくなってしまった.
\section{感想29114095-野竹浩二朗}
今回の課題は,前回作ったものを改良するというもので,どのように処理を高速するか悩んだ.自分で具体的な数字を与えるのではなく,なんとかプログラムで優先度を計算できないかと思ったが,計算方法が全く思いつかず,結果すべて手打ちで優先度を振るという無理やりな方法となってしまった.

% 参考文献
\begin{thebibliography}{99}
    \bibitem{swing} https://www.javadrive.jp/tutorial/
\end{thebibliography}

\end{document}
\documentclass[a4j]{jarticle}

\usepackage[dvipdfmx]{graphicx}
\usepackage{url}
\usepackage{listings,jlisting}
\usepackage{ascmac}
\usepackage{amsmath,amssymb}
\usepackage{here}

%ここからソースコードの表示に関する設定
\lstset{
  basicstyle={\ttfamily},
  identifierstyle={\small},
  commentstyle={\smallitshape},
  keywordstyle={\small\bfseries},
  ndkeywordstyle={\small},
  stringstyle={\small\ttfamily},
  frame={tb},
  breaklines=true,
  columns=[l]{fullflexible},
  numbers=left,
  xrightmargin=0zw,
  xleftmargin=3zw,
  numberstyle={\scriptsize},
  stepnumber=1,
  numbersep=1zw,
  lineskip=-0.5ex
}
%ここまでソースコードの表示に関する設定

\title{知能プログラミング演習II 課題6}
\author{グループ07\\
  29114086 飛世裕貴\\
%  {\small (グループレポートの場合は、グループ名および全員の学生番号と氏名が必要)}
}
\date{2019年1月5日}

\begin{document}
\maketitle

\paragraph{提出物} rep6
\paragraph{グループ} グループ07
\paragraph{メンバー}
\begin{tabular}{|c|c|c|}
  \hline\hline
  学籍番号&名前&貢献度\\
  \hline\hline
  29114007&池口弘尚&\\
  \hline
  29114031&大原拓人&\\
  \hline
  29114048&北原太一&\\
  \hline
  29114086&飛世裕貴&\\
  \hline
  29114095&野竹浩二朗&\\
  \hline
\end{tabular}



\section{課題の説明}
\begin{description}
\item[課題6-1] 課題5で実装したプランニングのプログラムの完成度を高めよ.課題5にやり残した発展課題があれば参考にして拡張しても良いし,全く新しい独自仕様を考案しても構わない.自由に拡張するか,あるいはもし残っていた問題点があれば完成度を高めよ.

\end{description}
私はオペレータ戦略の競合解消戦略のさらなる効率化を行った.
\section{課題6-1}
\begin{screen}
課題5で実装したプランニングのプログラムの完成度を高めよ.課題5にやり残した発展課題があれば参考にして拡張しても良いし,全く新しい独自仕様を考案しても構わない.自由に拡張するか,あるいはもし残っていた問題点があれば完成度を高めよ.
\end{screen}

\subsection{手法}
前回の課題ではオペレータの優先度を固定しており,その優先度順に展開するオペレータ順を決定していた.そのため,展開順は常に同じ順番になっていた.
ここではオペレータの優先度を固定するのではなく,その直前に選択したオペレータによって優先度を与えることで展開順を変更することで,処理時間を高速化する.

\subsection{実装}
直前に選択したオペレータによって優先度を与えるようにするため,以下のメソッドをPlanner.javaに追加した.
\begin{itemize}
\item judgeName:直前のオペレータは何かを判別する
\item resetPriority:優先度を再設定する
\end{itemize}
また,優先度をどのように与えたかを見やすくするため,配列priorityを作り,OPERATOR1から順に優先度を格納した.\\
judgeNameメソッドを以下に示す.
\begin{lstlisting}
public int judgeName(Operator op){
	if((op.name).contains("Place")){
		return 1;
	}else if((op.name).contains("remove")){
		return 2;
	}else if((op.name).contains("pick")){
		return 3;
	}else if((op.name).contains("put")){
		return 4;
	}else{
		return 0;
	}
}
\end{lstlisting}
上記したようにjudgeNameメソッドではオペレータの名前を取得し,含まれる単語でオペレータを判別している.なおinitOperatorsではオペレータを12個生成しているが,実際に積み木を操作するオペレータは4個であるため,その4個のみを判定している.\\
次にresetPriorityメソッドを以下に示す.
\begin{lstlisting}
void resetPriority(int judgeOperator){
	if(judgeOperator == 1){
		priority[0] = 7;
		priority[1] = 7;
		priority[2] = 2;
		priority[3] = 1;
	}else if(judgeOperator == 2){
		priority[0] = 7;
		priority[1] = 7;
		priority[2] = 1;
		priority[3] = 2;
	}else if(judgeOperator == 3){
		priority[0] = 2;
		priority[1] = 2;
		priority[2] = 7;
		priority[3] = 1;
	}else if(judgeOperator == 4){
		priority[0] = 2;
		priority[1] = 3;
		priority[2] = 1;
		priority[3] = 7;
	}
}
\end{lstlisting}
本プログラムではjudgeNameの返り値を取得し,その値を上記したresetPriorityメソッドの引数とすることで優先度を再設定している.
\section{実行結果}
ここでは改良した解消戦略を用いたプランニングを30回実行し,その平均実行時間を示す.また,性能の差を示すため前回実装した解消戦略を利用した際の平均時間も共に示す.
\begin{table}[H]
\begin{center}
\begin{tabular}{|l|c|}
\hline
戦略 & 平均実行時間(ミリ秒) \\ 
\hline
前回の解消戦略 & 1098 \\ \hline
今回の解消戦略 & 921 \\ \hline
\end{tabular}
\end{center}
\end{table}
この結果より,平均時間が約16%短くなっていることがわかった.
\section{考察}
上記したように,直前に選択したオペレータによって優先度を再設定しながらプランニングを実行した方がより高速に処理ができることがわかった.しかし,前回も取り上げたプログラマー依存性の問題は解消することができなかった.この問題の改善方法として,オペレータの出現頻度を用いる方法があげられる.各オペレータの出現頻度を正規化し,その値の大小関係にしたがって優先度を与える.このようにすることでプログラマが優先度を設定することなく,プランニングが可能になると考えられる.またこの方法に加えて,今回のように直前に選択されたオペレータごとに初期値を設定することで,プログラマの既存知識を利用することも可能になると考える.
\section{感想}
今回の課題では,実装はしなかったもののどのようにすればより良い優先度を与えられるかということを考えることが多かった.前回と今回の課題を通して,どのようにプログラマ依存を少なくするかということを考えていたが,プログラマ依存は必ずしも無くすべきものではないのではないかと感じた.考察の最後に記述したように既存の知識を利用できるようにした方が,より高速に処理を終えることができる場合は存在するだろう.そのため,今回のような課題では最初に優先度を設定し,その影響度合いを調整していくという方が有効であると考える.

このような影響度合いの調整は学習におけるパラメータ更新に通ずるものと言える.今後は学習のような数学モデルについても学んでいきたいと思う.
\end{document}
\documentclass[a4j]{jarticle}

\usepackage[dvipdfmx]{graphicx}
\usepackage{url}
\usepackage{listings,jlisting}
\usepackage{ascmac}
\usepackage{amsmath,amssymb}

%ここからソースコードの表示に関する設定
\lstset{
  basicstyle={\ttfamily},
  identifierstyle={\small},
  commentstyle={\smallitshape},
  keywordstyle={\small\bfseries},
  ndkeywordstyle={\small},
  stringstyle={\small\ttfamily},
  frame={tb},
  breaklines=true,
  columns=[l]{fullflexible},
  numbers=left,
  xrightmargin=0zw,
  xleftmargin=3zw,
  numberstyle={\scriptsize},
  stepnumber=1,
  numbersep=1zw,
  lineskip=-0.5ex
}
%ここまでソースコードの表示に関する設定

\title{知能プログラミング演習II 課題5}
\author{グループ07\\
  29114086 飛世裕貴\\
%  {\small (グループレポートの場合は、グループ名および全員の学生番号と氏名が必要)}
}
\date{2019年11月25日}

\begin{document}
\maketitle

\paragraph{提出物} rep5
\paragraph{グループ} グループ07
\paragraph{メンバー}
\begin{tabular}{|c|c|c|}
  \hline\hline
  学籍番号&名前&貢献度\\
  \hline\hline
  29114007&池口弘尚&\\
  \hline
  29114031&大原拓人&\\
  \hline
  29114048&北原太一&\\
  \hline
  29114086&飛世裕貴&\\
  \hline
  29114095&野竹浩二朗&\\
  \hline
\end{tabular}



\section{課題の説明}
\begin{description}
\item[課題5-1] 目標集合を変えてみたときに,動作が正しくない場合があったかどうか,実行例を示して考察せよ.
また,もしあったならその箇所を修正し,どのように修正したか記せ.

\item[課題5-2] 教科書のプログラムでは,オペレータ間の競合解消戦略としてランダムなオペレータ選択を採用している.
これを,効果的な競合解消戦略に改良すべく考察し,実装せよ.
改良の結果,性能がどの程度向上したかを定量的に(つまり数字で)示すこと.

\item[課題5-3] 上記のプランニングのプログラムでは,ブロックの属性(たとえば色や形など)を考えていないので,色や形などの属性を扱えるようにせよ.ルールとして表現すること.
例えば色と形の両方を扱えるようにする場合,Aが青い三角形,Bが黄色の四角形,Cが緑の台形であったとする.
その時,色と形を使ってもゴールを指定できるようにする("green on blue" や"blue on box"のように)

\item[課題5-4]上記5-2, 5-3で改良したプランニングシステムのGUIを実装せよ.
ブロック操作の過程をグラフィカルに可視化し,初期状態や目標状態をGUI上で変更できることが望ましい.

\end{description}

今回、私が担当した課題は課題5-2である。

\section{課題5-2}
\begin{screen}
教科書のプログラムでは,オペレータ間の競合解消戦略としてランダムなオペレータ選択を採用している.
これを,効果的な競合解消戦略に改良すべく考察し,実装せよ.
改良の結果,性能がどの程度向上したかを定量的に(つまり数字で)示すこと.
\end{screen}

\subsection{手法}
ここでは競合解消戦略として各オペレータに優先度を設ける事にした.優先度はOperatorクラスのインスタンスpriorityとして設けている.このpriorityの値により,考えられるオペレータ群をソートし最も優先度の高いものを選択するようにする.なおこの時,オペレータAを展開した後に再びオペレータAを展開してしまうことを許容してしまうと無限ループに陥る可能性が高くなるので,これを防ぐために展開を禁止するオペレータ群を設け,展開済みのオペレータを連続で展開するという動作を防ぐようにした.

\subsection{実装}
前述した手法を実装したコードを以下に示す.

\begin{lstlisting}[caption=PlannerクラスのplanningAGoalメソッドより変更点を抜粋]
sortOp();
Vector<String> theBlockNext = blockNext;

loop: for (int i = cPoint; i < operators.size(); i++) {
	step++;
	for (String dontnext : blockNext) {
		if (operators.elementAt(i).name.equals(dontnext))
			continue loop;
	}
	Operator anOperator = rename(operators.elementAt(i));
	
				   ・
				   ・
				(中略)
				   ・
				   ・
				   
	Vector<String> addList = anOperator.getAddList();
	for (int j = 0; j < addList.size(); j++) {
		if ((new Unifier()).unify(theGoal,
				addList.elementAt(j),
				theBinding)) {
			Operator newOperator = anOperator.instantiate(theBinding);
			for (String block : theBlockList) {
				if ((new Unifier()).unify(newOperator.name, block, theBinding))
					continue loop;
			}
			blockNext = anOperator.getBlockList();
			Vector<String> newGoals = newOperator.getIfList();
			Vector<String> addblocks = getBlockList(newGoals);
			theBlockList.addAll(addblocks);
			System.out.println(newOperator.name);
			if (planning(newGoals, theCurrentState, theBinding, theBlockList)) {
				newOperator = newOperator.instantiate(theBinding);
				System.out.println(newOperator.name);
				plan.addElement(newOperator);
				theCurrentState = newOperator.applyState(theCurrentState);
				theBlockList.removeAll(addblocks);
				blockNext.removeAllElements();
				return i + 1;
			} else {
			
				   ・
				   ・
				(後略)

\end{lstlisting}
\begin{verbatim}

\end{verbatim}
このコードにおいて,1行目のsortOpメソッドでは考えられるオペレータ群に対して優先度によってソートを行なっている.その後6〜9行目で選んだオペレータが禁止オペレータ群に含まれているかどうかを判定し,含まれているのであればその後の処理を飛ばして次のオペレータに関して同様の判定を行うようにしている.そして28〜31行目において,theBlockListにオペレータの展開を行うたびに禁止オペレータ群を加えていくことで前述したような同じオペレータを連続で行うという動作を防ぐようにした.

\subsection{実行例}
このプログラムを実行した結果,得られたプランを以下に示す.
\begin{screen}
\begin{verbatim}
*** GOALS ***[triangle on B, B on green]
\end{verbatim}
\vdots
(中略)\\
\vdots
\begin{verbatim}
***** This is a plan! *****
pick up A from the table
Place A on B
A is shape of triangle
B is shape of rect
triangle is a shape of A
rect is a shape of B
A is name of red
B is name of blue
red is a characteristic of A
blue is a characteristic of B
remove A from on top B
put A down on the table
pick up B from the table
Place B on C
C is name of green
pick up A from the table
Place A on B
A is shape of triangle
\end{verbatim}
\end{screen}
\begin{center}
図1:プランニング実行結果
\end{center}
\begin{verbatim}

\end{verbatim}
このようにプランが正しく得られている事が確認できた.またこの時,競合解消戦略としてランダムなオペレータ選択を行なった場合と優先度を用いてオペレータ選択を行った場合に各処理にかかるステップ数を以下に示す.
\begin{verbatim}

\end{verbatim}
\begin{center}
\begin{tabular}{|c|c|c|}
  \hline
  戦略&ステップ数\\
  \hline
  ランダム(100回平均)&14010.7\\
  \hline
  優先度&1099\\
  \hline
\end{tabular}
\end{center}
\begin{center}
表1:各戦略のステップ数
\end{center}
\begin{verbatim}

\end{verbatim}
このように優先度を用いて競合解消を行う事でランダムにオペレータ選択を行った場合に比べて,ステップ数が平均的に1/10程度に減少することが確認できた.

\subsection{考察}
今回の課題では上述したような実行結果が得られ,ランダムにオペレータ選択を行うよりも優先度を用いてオペレータ選択を行う方がよりステップ数が少なくなり,高速にプランニングが実行できることがわかった.しかし優先度を用いる場合には,その優先度をどのように決定するのか,よりステップ数を少なくする優先度の決め方はどのようなものかと言った問題点が生じてしまう.この問題はどのようにして理想的なヒューリスティックス関数を求めるかという問題と同様のもので,解決困難な問題である.また今回のように優先度をプログラマが決定するようにすると,得られる結果はプログラマ依存性の高いものになってしまうと考えられる.

\section{感想}

\end{document}

\documentclass{jarticle}

\usepackage{graphicx}
\usepackage{url}
\usepackage{listings,jlisting}
\usepackage{ascmac}
\usepackage{amsmath,amssymb}

%ここからソースコードの表示に関する設定
\lstset{
  basicstyle={\ttfamily},
  identifierstyle={\small},
  commentstyle={\smallitshape},
  keywordstyle={\small\bfseries},
  ndkeywordstyle={\small},
  stringstyle={\small\ttfamily},
  frame={tb},
  breaklines=true,
  columns=[l]{fullflexible},
  numbers=left,
  xrightmargin=0zw,
  xleftmargin=3zw,
  numberstyle={\scriptsize},
  stepnumber=1,
  numbersep=1zw,
  lineskip=-0.5ex
}
%ここまでソースコードの表示に関する設定

\title{知能プログラミング演習II 課題6}
\author{グループ7\\
  29114048 北原 太一\\
%  {\small (グループレポートの場合は、グループ名および全員の学生番号と氏名が必要)}
}
\date{2019年12月9日}

\begin{document}
\maketitle

\subparagraph{提出物} rep6
\subparagraph{グループ} グループ7
\subparagraph{メンバー}
\begin{tabular}{|c|c|c|}
  \hline
  学生番号&氏名&貢献度比率\\
  \hline\hline
  29114007&池口弘尚&100\\
  \hline
  29114031&大原拓人&100\\
  \hline
  29114048&北原太一&100\\
  \hline
  29114086&飛世裕貴&100\\
  \hline
  29114095&野竹浩二朗&100\\
  \hline
\end{tabular}

\section{課題の説明}
\begin{itembox}{課題6-1}
 課題5にやり残した発展課題があれば参考にして拡張しても良いし,全く新しい独自仕様を考案しても構わない.自由に拡張するか,あるいはもし残っていた問題点があれば完成度を高めよ.
\end{itembox}

私は、目標状態をソートするメソッドsortGoalListの作成を担当した.
\section{課題5の問題点}
課題5のプログラムにおいて,ソースコード\ref{lst:initGoal}のような目標集合を考える.
\newpage
  \begin{lstlisting}[caption=Planner.java(一部抜粋),label=lst:initGoal]
private Vector<String> initGoalList() {
  Vector<String> goalList = new Vector<String>();
  goalList.addElement("A on B");
  goalList.addElement("B on C");
  return goalList;
}
  \end{lstlisting}
 
この状態で実行すると,
\begin{enumerate}
\item[手順1] AをBの上に乗せる
\item[手順2] AをBの上から降ろす
\item[手順3] BをCの上に乗せる
\item[手順4] AをBの上に乗せる
\end{enumerate}
以上のような手順で問題を解決し,手順1, 2は無駄な操作となってしまう.状態集合にアサーションを追加する順番を入れ替えれば無駄な手順がなくなるため,我々は追加されたアサーションの順番をソートすることにした.

\section{実装}
実装に際して,directPriorityというHashTableと,indirectPriorityというVectorを考える.

directPriorityは,ブロックとその階数を対応付けている.例えば,テーブルの上にA,Aの上にB,Bの上にCがある場合,Aの階数は0,Bの階数は1,Cの階数は2である.

indirectPriorityは,ブロックとその下にあるブロックの対応を格納している.先述の例に従うと,indirectPriorityには\{B, A\}と\{C, B\}が格納される.
  
sortGoalListメソッドにおいては,まず目標集合に含まれるすべてのアサーションに対し,それが``ontable X''ならばdirectPriorityにXの階数が0であるという対応を追加し,``X on Y''ならばindirectPriorityに\{X, Y\}を追加する.その後,後述する方法でindirectPriorityをdirectPriorityに変換し,階数の少ないブロックに言及するアサーションが最初になるように目標状態をソートする.

\section{indirectPriorityからdirectPriorityへの変換}
変換は,以下のような手順で行う.
\begin{enumerate}
\item indirectPriorityから要素を1つ取り出す
\item その要素の2つ目のブロック,すなわち,下側にあるブロックの階数がすでにdirectPriorityからわかる場合,1つ目のブロック,すなわち,上側にあるブロックと(下側のブロックの階数+1)の対応をdirectPriorityに格納する
\item 2.の条件が満たされなかった場合,下側にあるブロックの階数を目標集合の要素数,上側にあるブロックの階数をその数+1とし,対応をdirectPriorityに格納する
\end{enumerate}

この手順をindirectPriorityの要素がなくなるまで繰り返すことにより,すべてのブロックの階数が分かる.

\section{考察}
sortGoalListメソッドをinitGoalListメソッドの直後に呼び出すことにより,効率よく目標アサーションの選択を行うことができるようになった.しかしながら,このメソッド中に二重以上のfor文が3回使われており,計算時間は長くなってしまっているため,プランニング全体の効率はあまり向上したとは言えない.
\end{document}
